\documentclass{beamer}
\usepackage[utf8]{inputenc}
\usepackage[czech]{babel}
\usepackage[T1]{fontenc}

\usetheme{Boadilla}
\usecolortheme{dolphin}

\title[tracer]{Determine applications affected by upgrade}
\author{Jakub Kadlčík}
\institute[UP]{Univerzita Palackého v Olomouci}
\date{7.~10.~2014}

\usepackage{listings}
\usepackage{listings}
\usepackage{color}

\definecolor{solarized@base03}{HTML}{002B36}
\definecolor{solarized@base02}{HTML}{073642}
\definecolor{solarized@base01}{HTML}{586e75}
\definecolor{solarized@base00}{HTML}{657b83}
\definecolor{solarized@base0}{HTML}{839496}
\definecolor{solarized@base1}{HTML}{93a1a1}
\definecolor{solarized@base2}{HTML}{EEE8D5}
\definecolor{solarized@base3}{HTML}{FDF6E3}
\definecolor{solarized@yellow}{HTML}{B58900}
\definecolor{solarized@orange}{HTML}{CB4B16}
\definecolor{solarized@red}{HTML}{DC322F}
\definecolor{solarized@magenta}{HTML}{D33682}
\definecolor{solarized@violet}{HTML}{6C71C4}
\definecolor{solarized@blue}{HTML}{268BD2}
\definecolor{solarized@cyan}{HTML}{2AA198}
\definecolor{solarized@green}{HTML}{859900}

\lstdefinestyle{solarized-light}
{
	basicstyle=\footnotesize\ttfamily\color{solarized@base00},
	backgroundcolor=\color{solarized@base3},
	rulesepcolor=\color{solarized@base3},
	numberstyle=\tiny\color{solarized@base1},
	keywordstyle=\color{solarized@green},
	stringstyle=\color{solarized@cyan}\ttfamily,
	identifierstyle=\color{solarized@blue},
	commentstyle=\color{solarized@base1},
	emphstyle=\color{solarized@red},
}

\lstdefinestyle{solarized-dark}
{
	basicstyle=\footnotesize\ttfamily\color{solarized@base01},
	backgroundcolor=\color{solarized@base03},
	rulesepcolor=\color{solarized@base03},
	numberstyle=\tiny\color{solarized@base01},
	keywordstyle=\color{solarized@green},
	stringstyle=\color{solarized@cyan}\ttfamily,
	identifierstyle=\color{solarized@base0},
	commentstyle=\color{solarized@base01},
	emphstyle=\color{solarized@red},
}

\lstset
{
	basicstyle=\footnotesize,        % Styl a typ písma
	captionpos=b,                    % Pozice popisku
	showstringspaces=false,          % Když true, místo mezer se vypíše podtržítko. e.g. "Hello_world"
	tabsize=4,                       % Velikost tabulátoru (počet mezer)
	style=solarized-dark,
	literate=
		{á}{{\'a}}1     {í}{{\'i}}1     {é}{{\'e}}1
		{ý}{{\'y}}1     {ú}{{\'u}}1     {ó}{{\'o}}1
		{ě}{{\v{e}}}1   {š}{{\v{s}}}1   {č}{{\v{c}}}1
		{ř}{{\v{r}}}1   {ž}{{\v{z}}}1   {ď}{{\v{d}}}1
		{ť}{{\v{t}}}1   {ň}{{\v{n}}}1   {ů}{{\r{u}}}1
		{Á}{{\'A}}1     {Í}{{\'I}}1     {É}{{\'E}}1
		{Ý}{{\'Y}}1     {Ú}{{\'U}}1     {Ó}{{\'O}}1
		{Ě}{{\v{E}}}1   {Š}{{\v{S}}}1   {Č}{{\v{C}}}1
		{Ř}{{\v{R}}}1   {Ž}{{\v{Z}}}1   {Ď}{{\v{D}}}1
		{Ť}{{\v{T}}}1   {Ň}{{\v{N}}}1   {Ů}{{\r{U}}}1
	,
}


\newcommand{\separator}{\vspace{15pt}}

\begin{document}

	\begin{frame}
		\titlepage
	\end{frame}

	\begin{frame}{Správa software v GNU/Linux}
		\begin{itemize}
			\item Instalace aplikací z balíčků
			\item Repozitáře s balíčky
			\item Spousta distribucí
			\item Různé typy balíčků (RPM, deb, \dots)
			\item Balíčkovací systémy (DNF, YUM, APT, \dots)
			\item Pravidelné aktualizace
		\end{itemize}
	\end{frame}

	\begin{frame}{Problém}
		Scénář:
		\begin{enumerate}
			\item Uživatel spustí aplikaci
			\item Aplikace načte do paměti všechny soubory potřebné k běhu
			\item Aplikace nadále pracuje se soubory v paměti
			\item \dots
			\item Uživatel spustí aktualizaci systému, která aplikaci ovlivní
			\item Spuštěná aplikace změnu nepocítí -- je potřeba ji restartovat
		\end{enumerate}
	\end{frame}

	\begin{frame}{Důsledky}
		\begin{itemize}
			\item Zpoždění nové funkcionality
			\item Zpoždění záplat chyb
			\item Problémy s konfigurací
			\item Nejistota opětovného startu aplikace
		\end{itemize}
	\end{frame}

	\begin{frame}{Zadání}
		Požadavky:
		\begin{itemize}
			\item Nalezení aktualizacemi ovlivněných aplikací
			\item Vypsání nápovědy, jak aplikace restartovat
		\end{itemize}

		\begin{itemize}
			\item Podpora distribuce Fedora
			\item Textové uživatelské rozhraní
			\item Plugin pro balíčkovací systém DNF
			\item Implementace v jazyce Python
		\end{itemize}
	\end{frame}

	\begin{frame}{Vyhledání ovlivněných aplikací}
		Algoritmus:
		\begin{enumerate}
			\item Získáme seznam všech balíčků modifikovaných od spuštění systému
			\item Získáme seznam všech spuštěných procesů
			\item Pro každý balíček a každý proces uděláme množinový průnik jejich souborů
			\item Neprázdný průnik znamená, že proces byl daným balíčkem ovlivněný
		\end{enumerate}
		\separator
		\begin{itemize}
			\item Nepříznivá časová složitost
		\end{itemize}
	\end{frame}

	% \begin{frame}{Vyhledání ovlivněných aplikací - Vylepšení}
	% 	Úpravy:
	% 	\begin{itemize}
	% 		\item Mnoho procesů využívá stejné soubory (knihovny, etc)
	% 		\item Nebudeme zkoumat proces po procesu, ale sjednocení jejich souborů
	% 		\item Struktura \texttt{\{ soubor1 : [proces1, proces2, \dots], \dots \}}
	% 	\end{itemize}
	% 	\separator
	% 	Vylepšený algoritmus:
	% 	\begin{enumerate}
	% 		\item Získáme seznam všech balíčků modifikovaných od spuštění systému
	% 		\item Získáme strukturu souborů a procesů
	% 		\item Každý soubor poskytovaný balíčkem hledáme ve struktuře
	% 		\item Nalezená shoda znamená, že balíček ovlivnil nalezené procesy
	% 	\end{enumerate}
	% \end{frame}

	\begin{frame}{Ukázka}
		\lstinputlisting{sources/tracer}
	\end{frame}

	\begin{frame}{Ukázka - Informace o aplikaci}
		\lstinputlisting{sources/helper}
	\end{frame}

	\begin{frame}{Závěr}
		Odkazy:
		\begin{itemize}
			\item http://tracer-package.com/
			\item http://docs.tracer-package.com/
			\item https://github.com/FrostyX/tracer
			\item https://admin.fedoraproject.org/pkgdb/package/tracer/
		\end{itemize}
		\vspace{70pt}
		\centerline{\textbf{Děkuji za pozornost.} Máte nějaké dotazy?}
	\end{frame}

\end{document}
