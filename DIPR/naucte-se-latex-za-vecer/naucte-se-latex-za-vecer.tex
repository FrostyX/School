\documentclass[10pt,a4paper]{article}
\usepackage[utf8]{inputenc}
\usepackage[czech]{babel}
\usepackage[T1]{fontenc}

\usepackage{amsfonts}
\usepackage{amsthm}
\newtheorem{definition}{Definice}
\usepackage[font=footnotesize]{subcaption}
\usepackage[colorlinks]{hyperref}
\hypersetup{citecolor=black}
\hypersetup{linkcolor=black}
\hypersetup{urlcolor=blue}


\newcommand{\inlinecode}{\texttt}

\title{Naučte se \LaTeX{} za večer}
\author{Jakub Kadlčík}
\date{9. 11. 2013}

\begin{document}
	\maketitle
	\newpage

	\section*{Předmluva}
	Tento dokument je určen každému, kdo se chce s co nejmenší námahou naučit používat typografický systém LaTeX. Nepředpokládá téměř žádné dovednosti, takže nezáleží na tom, zda jste žákem základní školy, studentem, který právě zjistil, že LaTeX není xy materiál a má v něm do konce týdne napsat práci, nebo důchodce, který chce, aby jeho písemné stížnosti vypadaly úhledně.
	Přesto je mířený na určitou cílovou skupinu lidí -- pro ty, kteří se chtějí naučit stěžejní věci v co nejkratším možném čase. Tedy aby mohli v LaTeXu psát pokud možno ihned a to i přesto, že se nedozvědí, xy.

	\newpage
	\tableofcontents
	\newpage

    \section{LaTeX?}
		\subsection{Úvod}
		Mluvíme-li o LaTeXu myslíme tím typografický systém sloužící k profesionální sazbě textů. Hlavním rozdílem mezi ním a textovými procesory, jakými jsou například MS Word, nebo OpenOffice Writer je ten, že dokument LaTeXu needitujete v takové podobě, v jaké bude tištěn. Pro formátování textu se používají různé příkazy a symboly -- dal by se tak přirovnat například k jazyku HTML.

		\subsection{Historie}
		Většina knih začíná mnoha stranami popisujícími detailní historii projektu. Znalost historie je zajisté důležitá, ale pokud chceme rychle začít něco dělat, je nám k ničemu. Zmínil bych jen základní fakta. LaTeX je sada maker pro systém TeX, který v 70. letech 20. století vytvořil profesor D. E. Knuth. Nebyl moc spokojený s tím, jak nakladatelství na  Stanfordově univerzitě sázelo jeho skripta do matematiky.

	\newpage
	\section{Instalace}
		LaTeX je k dispozici v různých distribucích - tou nejrozšířenější je TeX Live\footnote{\url{http://www.tug.org/texlive/}}. Nemusíte mít obavy, zda na vašem počítači bude fungovat. Podporuje totiž mnoho operačních systémů a architektur\footnote{\url{http://www.tug.org/texlive/doc.html}}.
		\\
		TeX Live vám sice umožní překlad zdrojových souborů na PDF dokumenty, ale o uživatelské přívětivosti nelze ani hovořit. Proto na scénu přichází Texmaker\footnote{\url{http://www.xm1math.net/texmaker/}}.

		\subsection{Linux}
		Jsem přesvědčen, že v každé rozumné linuxové distribuci stačí nainstalovat program Texmaker a díky závislostem se vám do systému dostane vše co k psaní v LaTeXu potřebujete. To je přece to, co na linuxu milujeme.

		\subsection{Windows a ostatní systémy}
		Pro instalaci distribuce TeX Live následujte \url{http://www.tug.org/texlive/doc.html}. Pro instalaci Texmakeru potom \url{http://www.xm1math.net/texmaker/download.html}.


	\newpage
	\section{Jedeme na to}
		V případě, že nejste příliš zdatní v českém jazyce, přečtěte si bonusovou kapitolu číslo X, pojednávající o české typografii. Vlastně, \dots{} Podívejte se tam raději všichni.

		\subsection{Dokument v LaTeXu}
		Jak jsem předeslal již v úvodu, soubor, se kterým člověk pracuje, vypadá jinak než výstupní soubor, který LaTeX vrací. Zdrojový kód je tedy kombinace formátovacích příkazů a textu. Tohle je naprosto stěžejní informace, pro jejíž ujasnění doporučuji srovnání tohoto dokumentu a jeho zdrojového kódu, který naleznete na adrese \url{http://frostyx.cz/posts/naucte-se-latex-za-vecer}.
		\\
		Teď už se pusťme do samotné tvorby zdrojového kódu.
		\subsection{Základní kostra dokumentu}
			Každý dokument psaný v LaTeXu musí obsahovat hlavičku a samotný text. Vždy když vytváříte nový dokument, automaticky do něj vložte následující kód:

			\begin{verbatim}
			\documentclass[10pt,a4paper]{article}
			\usepackage[utf8]{inputenc}
			\usepackage[T1]{fontenc}

			\begin{document}
			    Hello world
			\end{document}
			\end{verbatim}

			Obsah bloku ohraničeného příkazy \inlinecode{\textbackslash{}begin\{document\}} a \inlinecode{\textbackslash{}end\{document\}} je text výsledného dokumentu. Kód nad tímto blokem se nazývá hlavička. Ta slouží k nastavení velikosti papíru, kódování znaků, jazyku, atd.

			\subsubsection{Čeština}
			Při psaní českého dokumentu budete chtít, aby LaTeX zohledňoval české typografické speciality. Toho docílíte přidáním následujícího řádku do hlavičky.

			\begin{verbatim}
			\usepackage[czech]{babel}
			\end{verbatim}


		\subsection{Speciální znaky}
		Jak už jste si mohli všimnout, příkazy LaTeXu se skládají ze znaků \texttt{\textbackslash{}, \{, \}}, proto není možné je do textu jen tak zapsat, aby se nevyhodnotily na příkaz. Speciálních znaků však existuje mnohem více\footnote{Mnoho matematických symbolů lze najít na \url{http://web.ift.uib.no/Teori/KURS/WRK/TeX/symALL.html} a případě, že chcete zapsat symbol, pro nějž neznáte příkaz, stránka \url{http://detexify.kirelabs.org/classify.html} vám jistě pomůže.}, než jen tyto tři. Následující tabulka ukáže, jak je psát.

		\begin{tabular}{|p{4cm}|p{4cm}|}
			\hline
			\textbf{Znak}  & \textbf{Příkaz} \\ \hline
			\textbackslash & \textbackslash{}textbackslash \\ \hline
			\{             & \textbackslash\{ \\ \hline
			\}             & \textbackslash\} \\ \hline
			\#             & \textbackslash\# \\ \hline
			\$             & \textbackslash\$ \\ \hline
			\&             & \textbackslash\& \\ \hline
			\_             & \textbackslash\_ \\ \hline
			\%             & \textbackslash\% \\ \hline
		\end{tabular}
		
		
		\subsection{Strukturování textu}
			S psaním delších textů ruku v ruce přichází nutnost je členit na kapitoly, podkapitoly, odstavce, atd. Tedy je strukturovat. S klidem v srdci můžu říct, že v tomhle LaTeX sráží konkurenty na kolena.

			\paragraph{Kapitoly a odstavce}
				jsou řešeny pomocí několika málo příkazů.

				\begin{tabular}{|p{5cm}|p{6cm}|}
					\hline
					\textbackslash{}part             & Nová část \\ \hline
					\textbackslash{}section          & Nová sekce \\ \hline
					\textbackslash{}subsection       & Podsekce prvního řádu \\ \hline
					\textbackslash{}subsubsection    & Podsekce druhého řádu \\ \hline
					\textbackslash{}paragraph        & Nový odstavec \\ \hline
					\textbackslash{}subparagraph     & Nový pod-odstavec \\ \hline
					\textbackslash{}newpage          & Nová stránka \\ \hline
					\textbackslash{}\textbackslash{} & Nový řádek \\ \hline
				\end{tabular}

				Pomineme-li na okamžik poslední dva příkazy, pak každý z výše uvedených bere jako povinný argument název sekce -- například \texttt{\textbackslash{}section\{Úvod\}}. Navíc každý z nich existuje i ve variantě s hvězdičkou -- například \texttt{\textbackslash{}section*\{Úvod\}}. Taková sekce se potom nezobrazí v obsahu.

			\subsubsection{Odkaz na místo v textu}
			Libovolnou část dokumentu si můžeme označit příkazem \texttt{\textbackslash{}label\{návěští\}}. Poté se na místo můžeme odkázat dvěma způsoby. \texttt{\textbackslash{}ref\{návěští\}} vypíše číslo kapitoly a \texttt{\textbackslash{}pageref\{návěští\}} vypíše číslo stránky, kde se nachází odkazovaný text.

			\subsubsection{Obsah}
			Už žádné ruční psaní obsahu, nebo používání magie při jeho generování. Nikdy, nikdy nikdy. Prostě použijte příkaz \texttt{\textbackslash{}tableofcontents}.

		\subsection{Písmo}
			Následující tabulka č.X osvětlí, jak lze měnit formát písma.

			\begin{tabular}{|p{4cm}|p{5cm}|}
				\hline
				\texttt{\textbackslash{}textit} & \textit{kurzíva - italic} \\ \hline
				\texttt{\textbackslash{}textsf} & \textsf{skloněné - slanted} \\ \hline
				\texttt{\textbackslash{}texttt} & \texttt{strojopisné - typewritter} \\ \hline
				\texttt{\textbackslash{}textbf} & \textbf{tučné - bold face} \\ \hline
				\texttt{\textbackslash{}textsc} & \textsc{kapitálky - small caps} \\ \hline
			\end{tabular}

			Následující výčet ukáže, jak lze měnit jeho velikost (výčet je seřazen od nejmenšího po největší).

		\begin{itemize}
			\item \texttt{\textbackslash{}tiny}
			\item \texttt{\textbackslash{}scriptsize}
			\item \texttt{\textbackslash{}footnotesize}
			\item \texttt{\textbackslash{}small}
			\item \texttt{\textbackslash{}normalsize}
			\item \texttt{\textbackslash{}large}
			\item \texttt{\textbackslash{}Large}
			\item \texttt{\textbackslash{}LARGE}
			\item \texttt{\textbackslash{}huge}
			\item \texttt{\textbackslash{}Huge}
			\item \texttt{\textbackslash{}HUGE}
		\end{itemize}

		Jak můžete vidět, mnoho názvů příkazů LaTeXu je velmi intuitivních.

		\newpage
		\subsection{Seznamy}
		Řekne-li se seznam (případně výčet), nejspíš ne úplně každý bude mít jasnou představu o čem je řeč. Při čtení tohoto zdrojového kódu jste si mohli seznamu všimnout například v kapitole číslo XY, pojednávající o velikosti písma.
		\\
		Seznam může být číslovaný, \uv{odrážkovaný}, nebo uvozený libovolným znakem. Každý z nich může obsahovat libovolně mnoho podseznamů. Teorie by stačila, teď si je ukážeme.

		\subsubsection{\uv{Odrážkovaný} seznam}
			\begin{figure}[!ht]
				\centering
				\begin{subfigure}{.45\textwidth}
					\centering
					\begin{verbatim}
						\begin{itemize}
						    \item První položka
						    \item Druhá položka
						    \begin{itemize}
							\item A
							\item B
						    \end{itemize}
						    \item Třetí položka
						\end{itemize}
					\end{verbatim}
				\end{subfigure}
				\vline
				\begin{subfigure}{.45\textwidth}
					\centering
					\begin{itemize}
						\item První položka
						\item Druhá položka
							\begin{itemize}
								\item A
								\item B
							\end{itemize}
						\item Třetí položka
					\end{itemize}
				\end{subfigure}
			\end{figure}

		\subsubsection{Číslovaný seznam}
			\begin{figure}[!ht]
				\centering
				\begin{subfigure}{.45\textwidth}
					\centering
					\begin{verbatim}
						\begin{enumerate}
						    \item První položka
						    \item Druhá položka
						    \begin{itemize}
							\item A
							\item B
						    \end{enumerate}
						    \item Třetí položka
						\end{itemize}
					\end{verbatim}
				\end{subfigure}
				\vline
				\begin{subfigure}{.45\textwidth}
					\centering
					\begin{enumerate}
						\item První položka
						\item Druhá položka
							\begin{itemize}
								\item A
								\item B
							\end{itemize}
						\item Třetí položka
					\end{enumerate}
				\end{subfigure}
			\end{figure}

		Nehledě na to, jaký typ seznamu se použije, jeho body vždy začínají příkazem \texttt{\textbackslash{}item}, takže změnit typ seznamu není vůbec komplikované. Navíc můžete vidět, že není problém udělat číslovaný seznam, který bude obsahovat \uv{odrážkovaný} podseznam.

		\newpage
		\subsection{Tabulky}
		Nemůžu říci, že se v LaTeXu vytvářejí tabulky tak jednoduše, jako v MS Word, nebo OpenOffice Writer, ale i tak se nejedná o nic složitého.
		
			\begin{figure}[!ht]
				\centering
				\begin{subfigure}{.45\textwidth}
					\centering
					\begin{verbatim}
						\begin{tabular}{|c|c|}
						    \hline
						    AAA & BBB \\\hline
						    CCC & DDD \\\hline
						    EEE & FFF \\\hline
						    GGG & HHH \\\hline
						\end{tabular}
					\end{verbatim}
				\end{subfigure}
				\vline
				\begin{subfigure}{.45\textwidth}
					\centering
					\begin{tabular}{|c|c|}
						\hline
						AAA & BBB \\\hline
						CCC & DDD \\\hline
						EEE & FFF \\\hline
						GGG & HHH \\\hline
					\end{tabular}
				\end{subfigure}
			\end{figure}
			
			Můžete vidět, že tabulku reprezentuje prostředí \texttt{tabular}, který přebírá jeden povinný argument. Pojďme si ho teď objasnit. Znak \texttt{|} reprezentuje, kde budou v tabulce svislé \uv{čáry}. Písmena mezi nimi reprezentují počet sloupců a zarovnání textu v nich. Lze použít \texttt{l} - zarovnání vlevo, \texttt{r} - zarovnání vpravo, \texttt{c} - zarovnání na střed.
			\\
			Nový řádek tabulky není uvozen žádným příkazem, jak je tomu u seznamů, naopak na konec každého řádku je potřeba napsat \texttt{\textbackslash{}\textbackslash}. Vodorovné \uv{čáry} za nás LaTeX bohužel taky sám nevypíše. Musíme to udělat příkazem \texttt{\textbackslash{}hline}.
			\\ A konečně znak \texttt{\&}. Určitě vám došlo, že odděluje sloupce.
				
		
		\subsection{Obrázky}
		
		\subsection{Matematika}
		Matematika -- důvod, proč LaTeX vznikl. Taky vlastnost, ve které bezkonkurenčně a nelítostně likviduje všechny své konkurenty. My si ukážeme pouze pár základních věcí, protože chceme tento dokument co nejkratší (v opačném případě by postrádal smysl a vy si rovnou můžete přečíst knihu). Zájemce o více matematiky bych rád odkázal zde \url{http://en.wikibooks.org/wiki/LaTeX/Mathematics}.
		\\
		\\
		Matematické prostředí lze spustit několika způsoby. Ukážeme si je na předpisu lineární funkce:
		\begin{enumerate}
			\item \texttt{\$f(x) = ax + b\$}
			\item \texttt{\$\$f(x) = ax + b\$\$	}
			\item \texttt{\textbackslash{}begin\{equation\} f(x) = ax + b \textbackslash{}end\{equation\}}
		\end{enumerate}

		Každá ze zmíněných možností vloží vzoreček jinak. První se hodí, když jej chceme vložit do textu, protože přesně to se stane. Například chci říci, že \uv{$f(x) = ax + y$ je předpis lineární funkce}. Druhá možnost vypíše vzoreček na samostatném řádku, třetí jej navíc očísluje.
		
		\subsubsection{Mocniny a odmocniny}		
			Na následující rovnici si ukážeme, jak psát mocniny a odmocniny:
			$$2^2 + 2^{2+1} = 8 + \sqrt[3]{8} + \sqrt{4}$$
			Vstup: \texttt{\$\$2\^{}2 + 2\^{}\{2+1\} = 8 + \textbackslash{}sqrt[3]\{8\} + \textbackslash{}sqrt\{4\}\$\$}
		
		\subsubsection{Indexy}
			Indexy si můžeme ukázat na řadě:
			$$x_1 + x_2 + x_3 + \dots + x_{n-2} + x_{n-1} + x_n $$
			Vstup: \texttt{\$\$x\_1 + x\_2 + x\_3 + \textbackslash{}dots + x\_\{n-2\} + x\_\{n-1\} + x\_n\$\$}
		
		\subsubsection{Zlomky}
			Poznáte tento vzorec? 
			$$x_{1,2} = \frac{-b \pm \sqrt{D}}{2a}$$
			Vstup: \texttt{\$\$x\_\{1,2\} = \textbackslash{}frac\{-b \textbackslash{}pm \textbackslash{}sqrt\{D\}\}\{2a\}\$\$}
		
		\subsubsection{Kombinační čísla}
			Umíte spočítat hodnotu kombinačního čísla? 		
			$${n \choose k} = \frac{n!}{(n-k)! \cdot k!}$$	
			Teď už určitě umíte.
			\\
			Vstup: \texttt{\$\$\{n \textbackslash{}choose k\} = \textbackslash{}frac\{n!\}\{(n-k)! \textbackslash{}cdot k!\}\$\$}
		
		\subsubsection{Závorky}
			Když číslo tři patří do intervalu 1 až 10, pak určitě patří mezi reálná čísla.
			$$3 \in \langle 1; 10\rangle \Rightarrow 3 \in \mathbb{R}$$
			Vstup: \texttt{\$\$3 \textbackslash{}in \textbackslash{}langle 1; 10\textbackslash{}rangle \textbackslash{}Rightarrow 3 \textbackslash{}in \textbackslash{}mathbb\{R\}\$\$}\\		
			\vspace{1cm}
			\\
			Na základní škole se často používá několik druhů závorek namísto kulatých. To aby byl pro žáky příklad přehlednější.
			$$10 \cdot \left\{\frac{10-2}{4} + \left[5 + 5 - \left(3 + 6 - 2\right)\right]\right\} =$$
			Vstup: \texttt{\$\$10 \textbackslash{}cdot \textbackslash{}left\textbackslash{}\{\textbackslash{}frac\{10-2\}\{4\} + \textbackslash{}left[5 + 5 - \textbackslash{}left(3 + 6 - 2\textbackslash{}right)\textbackslash{}right]\textbackslash{}right\textbackslash{}\} =\$\$}
		
		\subsubsection{Definice}
		Nejdříve je potřeba vložit do hlavičky řádek \texttt{\textbackslash{}usepackage\{amsthm\}} a (stále v hlavičce) pomocí příkazu \texttt{\textbackslash{}newtheorem\{definition\}\{Definice\}} seznámit LaTeX s novým prostředím \texttt{definition}. Potom už si můžeme napsat takhle krásnou definici.
		
		\begin{definition}
			Nechť $A \neq 0$. \textbf{Binární operací na množině} A nazveme každé zobrazení $f: A \times A \rightarrow A$.
		\end{definition}
		
		Jak že jsem to udělal? Podívejte se na kód:
		\begin{verbatim}
			\begin{definition}
			    Nechť $A \neq 0$. \textbf{Binární operací na množině} A
			    nazveme každé zobrazení $f: A \times A \rightarrow A$.
			\end{definition}
		\end{verbatim}
				
		\subsection{Seznam literatury}

	\newpage
	\section{Typografie}
\end{document}
